% Options for packages loaded elsewhere
% Options for packages loaded elsewhere
\PassOptionsToPackage{unicode}{hyperref}
\PassOptionsToPackage{hyphens}{url}
\PassOptionsToPackage{dvipsnames,svgnames,x11names}{xcolor}
%
\documentclass[
  letterpaper,
  DIV=11,
  numbers=noendperiod]{scrartcl}
\usepackage{xcolor}
\usepackage{amsmath,amssymb}
\setcounter{secnumdepth}{-\maxdimen} % remove section numbering
\usepackage{iftex}
\ifPDFTeX
  \usepackage[T1]{fontenc}
  \usepackage[utf8]{inputenc}
  \usepackage{textcomp} % provide euro and other symbols
\else % if luatex or xetex
  \usepackage{unicode-math} % this also loads fontspec
  \defaultfontfeatures{Scale=MatchLowercase}
  \defaultfontfeatures[\rmfamily]{Ligatures=TeX,Scale=1}
\fi
\usepackage{lmodern}
\ifPDFTeX\else
  % xetex/luatex font selection
\fi
% Use upquote if available, for straight quotes in verbatim environments
\IfFileExists{upquote.sty}{\usepackage{upquote}}{}
\IfFileExists{microtype.sty}{% use microtype if available
  \usepackage[]{microtype}
  \UseMicrotypeSet[protrusion]{basicmath} % disable protrusion for tt fonts
}{}
\makeatletter
\@ifundefined{KOMAClassName}{% if non-KOMA class
  \IfFileExists{parskip.sty}{%
    \usepackage{parskip}
  }{% else
    \setlength{\parindent}{0pt}
    \setlength{\parskip}{6pt plus 2pt minus 1pt}}
}{% if KOMA class
  \KOMAoptions{parskip=half}}
\makeatother
% Make \paragraph and \subparagraph free-standing
\makeatletter
\ifx\paragraph\undefined\else
  \let\oldparagraph\paragraph
  \renewcommand{\paragraph}{
    \@ifstar
      \xxxParagraphStar
      \xxxParagraphNoStar
  }
  \newcommand{\xxxParagraphStar}[1]{\oldparagraph*{#1}\mbox{}}
  \newcommand{\xxxParagraphNoStar}[1]{\oldparagraph{#1}\mbox{}}
\fi
\ifx\subparagraph\undefined\else
  \let\oldsubparagraph\subparagraph
  \renewcommand{\subparagraph}{
    \@ifstar
      \xxxSubParagraphStar
      \xxxSubParagraphNoStar
  }
  \newcommand{\xxxSubParagraphStar}[1]{\oldsubparagraph*{#1}\mbox{}}
  \newcommand{\xxxSubParagraphNoStar}[1]{\oldsubparagraph{#1}\mbox{}}
\fi
\makeatother


\usepackage{longtable,booktabs,array}
\usepackage{calc} % for calculating minipage widths
% Correct order of tables after \paragraph or \subparagraph
\usepackage{etoolbox}
\makeatletter
\patchcmd\longtable{\par}{\if@noskipsec\mbox{}\fi\par}{}{}
\makeatother
% Allow footnotes in longtable head/foot
\IfFileExists{footnotehyper.sty}{\usepackage{footnotehyper}}{\usepackage{footnote}}
\makesavenoteenv{longtable}
\usepackage{graphicx}
\makeatletter
\newsavebox\pandoc@box
\newcommand*\pandocbounded[1]{% scales image to fit in text height/width
  \sbox\pandoc@box{#1}%
  \Gscale@div\@tempa{\textheight}{\dimexpr\ht\pandoc@box+\dp\pandoc@box\relax}%
  \Gscale@div\@tempb{\linewidth}{\wd\pandoc@box}%
  \ifdim\@tempb\p@<\@tempa\p@\let\@tempa\@tempb\fi% select the smaller of both
  \ifdim\@tempa\p@<\p@\scalebox{\@tempa}{\usebox\pandoc@box}%
  \else\usebox{\pandoc@box}%
  \fi%
}
% Set default figure placement to htbp
\def\fps@figure{htbp}
\makeatother


% definitions for citeproc citations
\NewDocumentCommand\citeproctext{}{}
\NewDocumentCommand\citeproc{mm}{%
  \begingroup\def\citeproctext{#2}\cite{#1}\endgroup}
\makeatletter
 % allow citations to break across lines
 \let\@cite@ofmt\@firstofone
 % avoid brackets around text for \cite:
 \def\@biblabel#1{}
 \def\@cite#1#2{{#1\if@tempswa , #2\fi}}
\makeatother
\newlength{\cslhangindent}
\setlength{\cslhangindent}{1.5em}
\newlength{\csllabelwidth}
\setlength{\csllabelwidth}{3em}
\newenvironment{CSLReferences}[2] % #1 hanging-indent, #2 entry-spacing
 {\begin{list}{}{%
  \setlength{\itemindent}{0pt}
  \setlength{\leftmargin}{0pt}
  \setlength{\parsep}{0pt}
  % turn on hanging indent if param 1 is 1
  \ifodd #1
   \setlength{\leftmargin}{\cslhangindent}
   \setlength{\itemindent}{-1\cslhangindent}
  \fi
  % set entry spacing
  \setlength{\itemsep}{#2\baselineskip}}}
 {\end{list}}
\usepackage{calc}
\newcommand{\CSLBlock}[1]{\hfill\break\parbox[t]{\linewidth}{\strut\ignorespaces#1\strut}}
\newcommand{\CSLLeftMargin}[1]{\parbox[t]{\csllabelwidth}{\strut#1\strut}}
\newcommand{\CSLRightInline}[1]{\parbox[t]{\linewidth - \csllabelwidth}{\strut#1\strut}}
\newcommand{\CSLIndent}[1]{\hspace{\cslhangindent}#1}



\setlength{\emergencystretch}{3em} % prevent overfull lines

\providecommand{\tightlist}{%
  \setlength{\itemsep}{0pt}\setlength{\parskip}{0pt}}



 


\KOMAoption{captions}{tableheading}
\makeatletter
\@ifpackageloaded{tcolorbox}{}{\usepackage[skins,breakable]{tcolorbox}}
\@ifpackageloaded{fontawesome5}{}{\usepackage{fontawesome5}}
\definecolor{quarto-callout-color}{HTML}{909090}
\definecolor{quarto-callout-note-color}{HTML}{0758E5}
\definecolor{quarto-callout-important-color}{HTML}{CC1914}
\definecolor{quarto-callout-warning-color}{HTML}{EB9113}
\definecolor{quarto-callout-tip-color}{HTML}{00A047}
\definecolor{quarto-callout-caution-color}{HTML}{FC5300}
\definecolor{quarto-callout-color-frame}{HTML}{acacac}
\definecolor{quarto-callout-note-color-frame}{HTML}{4582ec}
\definecolor{quarto-callout-important-color-frame}{HTML}{d9534f}
\definecolor{quarto-callout-warning-color-frame}{HTML}{f0ad4e}
\definecolor{quarto-callout-tip-color-frame}{HTML}{02b875}
\definecolor{quarto-callout-caution-color-frame}{HTML}{fd7e14}
\makeatother
\makeatletter
\@ifpackageloaded{caption}{}{\usepackage{caption}}
\AtBeginDocument{%
\ifdefined\contentsname
  \renewcommand*\contentsname{Table of contents}
\else
  \newcommand\contentsname{Table of contents}
\fi
\ifdefined\listfigurename
  \renewcommand*\listfigurename{List of Figures}
\else
  \newcommand\listfigurename{List of Figures}
\fi
\ifdefined\listtablename
  \renewcommand*\listtablename{List of Tables}
\else
  \newcommand\listtablename{List of Tables}
\fi
\ifdefined\figurename
  \renewcommand*\figurename{Figure}
\else
  \newcommand\figurename{Figure}
\fi
\ifdefined\tablename
  \renewcommand*\tablename{Table}
\else
  \newcommand\tablename{Table}
\fi
}
\@ifpackageloaded{float}{}{\usepackage{float}}
\floatstyle{ruled}
\@ifundefined{c@chapter}{\newfloat{codelisting}{h}{lop}}{\newfloat{codelisting}{h}{lop}[chapter]}
\floatname{codelisting}{Listing}
\newcommand*\listoflistings{\listof{codelisting}{List of Listings}}
\makeatother
\makeatletter
\makeatother
\makeatletter
\@ifpackageloaded{caption}{}{\usepackage{caption}}
\@ifpackageloaded{subcaption}{}{\usepackage{subcaption}}
\makeatother
\usepackage{bookmark}
\IfFileExists{xurl.sty}{\usepackage{xurl}}{} % add URL line breaks if available
\urlstyle{same}
\hypersetup{
  pdftitle={Microsourcing: A literature review (illustration)},
  colorlinks=true,
  linkcolor={blue},
  filecolor={Maroon},
  citecolor={Blue},
  urlcolor={Blue},
  pdfcreator={LaTeX via pandoc}}


\title{Microsourcing: A literature review (illustration)}
\author{Gerit Wagner \and Julian Prester \and Roman Lukyanenko \and Guy
Paré}
\date{}
\begin{document}
\maketitle


\begin{tcolorbox}[enhanced jigsaw, toprule=.15mm, colbacktitle=quarto-callout-note-color!10!white, coltitle=black, title=\textcolor{quarto-callout-note-color}{\faInfo}\hspace{0.5em}{Note}, arc=.35mm, opacitybacktitle=0.6, colframe=quarto-callout-note-color-frame, titlerule=0mm, colback=white, bottomtitle=1mm, leftrule=.75mm, bottomrule=.15mm, left=2mm, toptitle=1mm, rightrule=.15mm, breakable, opacityback=0]

\begin{itemize}
\tightlist
\item
  Create colrev repository in parallel
\item
  Add this manuscript vignette at the end
\item
  Push to
  \href{https://github.com/fs-ise/C5-DM-vignette}{C5-DM-vignette}
\item
  Publish as Quarto website
\item
  Update links in this document
\item
  Include vignette (screenshot?) in the paper
\item
  Share with JP
\end{itemize}

\end{tcolorbox}

\subsection{Plan}\label{plan}

\begin{itemize}
\item
  The review is conducted using a
  \href{https://github.com/fs-ise/C5-DM-vignette}{shared GitHub
  repository}, which was be synchronized locally by the team
\item
  Record metadata is curated as follows:

  \begin{itemize}
  \tightlist
  \item
    Data retrieved in the search is stored in the
    \href{temp_file.txt}{data/raw} directory; the
    \href{temp_file.txt}{Git history of this path} shows that the files
    were preserved in their original form, i.e., treated as raw data
  \item
    Records were imported into the
    \href{temp_file.txt}{data/records.bib} as the primary data
    structure; the \href{temp_file.txt}{Git history of this file}
    documents how each record evolved through the process (e.g., manual
    or computational change of metadata, merging of records,
    prescreening decisions)
  \end{itemize}
\item
  For primary data (record metadata), the Bibtex format was chosen and
  consistent formatting was ensured by CoLRev. BibTex is machine
  readable and the changes can easily be interpreted when inspecting the
  git history.
\end{itemize}

\begin{figure}[H]

{\centering \pandocbounded{\includegraphics[keepaspectratio]{figures/recommendation.png}}

}

\caption{Data structures}

\end{figure}%

\subsection{Search}\label{search}

\begin{itemize}
\tightlist
\item
  The search strategy is stored together with the raw data files
  \href{temp_file.txt}{here}, in line with the standard of Haddaway et
  al. (2022).
\item
  Open-access API-searches (licensing issues permit publication of raw
  data exported from databases like WOS or EBSCO)
\item
  We reused samples from prior reviews (Wagner, Prester, and Paré 2021;
  Fiers 2023): \href{temp_file.txt}{hrere}
\end{itemize}

\begin{tcolorbox}[enhanced jigsaw, toprule=.15mm, colbacktitle=quarto-callout-note-color!10!white, coltitle=black, title=\textcolor{quarto-callout-note-color}{\faInfo}\hspace{0.5em}{Note}, arc=.35mm, opacitybacktitle=0.6, colframe=quarto-callout-note-color-frame, titlerule=0mm, colback=white, bottomtitle=1mm, leftrule=.75mm, bottomrule=.15mm, left=2mm, toptitle=1mm, rightrule=.15mm, breakable, opacityback=0]

\begin{itemize}
\tightlist
\item
  Search-query was used to validate syntactic correctness and \ldots{}
\item
  scope (????)
\end{itemize}

\end{tcolorbox}

\subsection{Dedupe}\label{dedupe}

\begin{itemize}
\tightlist
\item
  Preparation was done using CoLRev and extensions (see
  \href{temp_file.txt}{prep commit}).
\item
  Deduplication was done using BibDedupe. Deduplication changes are in
  \href{temp_file.txt}{dedupe commit}.
\end{itemize}

\begin{tcolorbox}[enhanced jigsaw, toprule=.15mm, colbacktitle=quarto-callout-note-color!10!white, coltitle=black, title=\textcolor{quarto-callout-note-color}{\faInfo}\hspace{0.5em}{Note}, arc=.35mm, opacitybacktitle=0.6, colframe=quarto-callout-note-color-frame, titlerule=0mm, colback=white, bottomtitle=1mm, leftrule=.75mm, bottomrule=.15mm, left=2mm, toptitle=1mm, rightrule=.15mm, breakable, opacityback=0]

Dedupe changes were validated using the max-diff strategy
(\texttt{colrev\ validate\ XXXX}). Preparation changes were validated
using the max-diff strategy (\texttt{colrev\ validate\ XXXX}).

\end{tcolorbox}

\subsection{Prescreen}\label{prescreen}

\begin{figure}[H]

{\centering \pandocbounded{\includegraphics[keepaspectratio]{figures/illustration-lr-transparency.png}}

}

\caption{Data structures}

\end{figure}%

\begin{tcolorbox}[enhanced jigsaw, toprule=.15mm, colbacktitle=quarto-callout-note-color!10!white, coltitle=black, title=\textcolor{quarto-callout-note-color}{\faInfo}\hspace{0.5em}{Note}, arc=.35mm, opacitybacktitle=0.6, colframe=quarto-callout-note-color-frame, titlerule=0mm, colback=white, bottomtitle=1mm, leftrule=.75mm, bottomrule=.15mm, left=2mm, toptitle=1mm, rightrule=.15mm, breakable, opacityback=0]

\begin{itemize}
\tightlist
\item
  For prescreening, we tested the new
  \href{temp_file.txt}{llm-prescreener} in \href{temp_file.txt}{ref}.
  Comparison with prescreening decisions of GW showed low reliability
  with the llm-prescreener (command + kappa). Results were therefore
  reverted (\href{temp_file.txt}{ref}) and a fully manual prescreen was
  implemented.
\end{itemize}

Note: this could also be done in a separate branch, or the changes could
be undone using a hard git reset.

\begin{itemize}
\tightlist
\item
  Screen: fulltext documents were shared in a protected drive (link to
  Dropbox)
\end{itemize}

\end{tcolorbox}

\subsection{Data extraction}\label{data-extraction}

\begin{itemize}
\item
  For data extraction, four scenarios were considered:

  \begin{itemize}
  \tightlist
  \item
    A purely narrative review (the synthesis is written in
    \href{temp_file.txt}{this} document)
  \item
    A bibliometric analysis (the citation network is
    \href{temp_file.txt}{here})
  \item
    An emergent mapping study (the notes are \href{temp_file.txt}{here}
    and illustrated \href{temp_file.txt}{here})
  \item
    A structured extraction of evidence (a preliminary coding scheme is
    \href{temp_file.txt}{here} and the pilot coding
    \href{temp_file.txt}{here})
  \end{itemize}
\end{itemize}

\begin{figure}[H]

{\centering \pandocbounded{\includegraphics[keepaspectratio]{figures/data_structures.png}}

}

\caption{Data structures}

\end{figure}%

\subsection{Synthesis}\label{synthesis}

\begin{itemize}
\item
  The narrative synthesis is in \href{temp_file.txt}{this} document in
  Markdown format, allowing for larger teams to work on the same
  document (similar to \emph{cite-covid-review})
\item
  To make the review reusable, we added the \href{temp_file.txt}{XY}
  license (indexing in SYNERGY, SearchRXiv is planned once the review
  progresses beyond the \emph{illustration} stages)
\item
  We reused prior reviews as follows:

  \begin{itemize}
  \tightlist
  \item
    The search strategy of XY were considered in the design of the
    database searches
  \item
    The sample of XY was imported as part of the search.
  \end{itemize}
\item
  The current status of the project is automatically updated with every
  change and reflected in the PRISMA chart (Page et al. 2021):
\end{itemize}

\begin{figure}[H]

{\centering \pandocbounded{\includegraphics[keepaspectratio]{prisma.png}}

}

\caption{PRISMA Flow Chart (generated by
\href{temp_file.txt}{python-prisma})}

\end{figure}%

\subsection{Repository}\label{repository}

\begin{itemize}
\tightlist
\item
  \href{https://github.com/digital-work-lab/project-name}{GitHub
  Repository}
\end{itemize}

\phantomsection\label{refs}
\begin{CSLReferences}{1}{0}
\bibitem[\citeproctext]{ref-Fiers2023}
Fiers, Fien. 2023. {``Inequality and Discrimination in the Online Labor
Market: A Scoping Review.''} \emph{New Media \& Society} 25 (12):
3714--34. \url{https://doi.org/10.1177/14614448221128379}.

\bibitem[\citeproctext]{ref-HaddawayRethlefsenDaviesEtAl2022}
Haddaway, Neal R., Melissa L. Rethlefsen, Melinda Davies, Julie
Glanville, Bethany McGowan, Kate Nyhan, and Sarah Young. 2022. {``A
Suggested Data Structure for Transparent and Repeatable Reporting of
Bibliographic Searching.''} \emph{Campbell Systematic Reviews} 18 (4):
1--12. \url{https://doi.org/10.1002/CL2.1288}.

\bibitem[\citeproctext]{ref-PageMcKenzieBossuytEtAl2021}
Page, Matthew J., Joanne E. McKenzie, Patrick M. Bossuyt, Isabelle
Boutron, Tammy C. Hoffmann, Cynthia D. Mulrow, Larissa Shamseer, et al.
2021. {``The PRISMA 2020 Statement: An Updated Guideline for Reporting
Systematic Reviews.''} \emph{Systematic Reviews} 10 (1).
\url{https://doi.org/10.1186/S13643-021-01626-4}.

\bibitem[\citeproctext]{ref-WagnerPresterPare2021}
Wagner, Gerit, Julian Prester, and Guy Paré. 2021. {``Exploring the
Boundaries and Processes of Digital Platforms for Knowledge Work: A
Review of Information Systems Research.''} \emph{The Journal of
Strategic Information Systems} 30 (4): 101694.
\url{https://doi.org/10.1016/j.jsis.2021.101694}.

\end{CSLReferences}




\end{document}
